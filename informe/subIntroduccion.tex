\section{Introducci\'on}
El presente informe resume el comportamiento y algoritmos utilizados para la creaci\'on de los filtros de Diferencia de im\'agenes y Blur Gaussiano en sus impelemtaciones en C y Ensamblador, los resultados de los experimentos que se realizaron para evaluar las diferencias entre ambas
implementaciones de los filtros de im\'agenes. As\'i como las conclusiones obtenidas y evidencias de qu\'e cambios podr\'ian mejorar el rendimiento en ciertos casos.

\subsection{Descripci\'on de metodolog\'ia de pruebas}
Las mediciones se realizar\'an mediante un script de python en el cual se ejecuta: \\
\recuadro{\begin{center}\textit{perf stat -B -e cache-references,cache-misses,cycles,instructions}\end{center}}
para cada imagen del conjunto de prueba y para cada par\'ametro de entrada. \\
Esto nos permite, adem\'as de los datos informados por el binario de la c\'atedra, evaluar cantidad de ciclos, instrucciones ejecutadas, estado de cache, etc.

Se planea ejecutar 500 iteraciones de los algoritmos correspondientes, evitando leer/abrir/escribir la imagen en cada iteraci\'on, lo que nos permite medir casi exclusivamente el tiempo de ejecuci\'on del algoritmo. \\
Para eliminar diferencias en los tiempos de ejecuci\'on debido a temas de clock se realizar\'an todas las pruebas en una misma computadora, utilizamdo el governor del procesador en \textbf{performance} y para descartar que procesos en segundo plano alteren los resultados los tests se ejecutar\'an con un par\'ametro \textbf{\textit{nice}} de $-20$. \\
En la secci\'on \ref{sec:specs} se detalla la informaci\'on de la computadora donde se corren los tests y sistema operativo de la misma. \\
Los tests de las implementaci\'on en \emph{C} se corrieron con el nivel de optimizaci\'on "\textbf{-O3}".

En cada prueba particular se definir\'an, si hay, los cambios en las metodolog\'ias de pruebas y se evidenciar\'a con salida de c\'odigo o con diferentes gr\'aficos cuales fueron los resultados obtenidos. \\
Ya que los algoritmos no dependen del contenido de la im\'agen se generaron imagenes aleatorias con la herramienta \textit{imagemagick}.
A partir de las im\'agenes aleatorias se crearon copias de distintos tama\~nos: 16x16 [pixeles], 32x32, 64x64, 256x256 y 1024x1024. \\
Adicionalmente, se crearon algunas im\'agenes de otras dimensiones para evaluar el efecto de ciertas caracter\'isticas en el ejecuci\'on de los algoritmos; por ejemplo, im\'agenes de 128x512 y 512x128 para determinar si la \'unica variable en el tiempo de ejecuci\'on es el tama\~no o si el ancho y el alto (manteniendo una cantidad constante de pixeles) tambi\'en lo afectan.

\subsection{Especificaci\'on t\'ecnica}
\label{sec:specs}
Todas las pruebas las corrimos en la misma computadora, con las siguientes especificaciones
\begin{itemize}
\item Procesador: AMD FX-8350
\item Memoria Ram: 8GB
\item Sistema Operativo: Debian Testing (Linux 4.1.0-2)
\item Governor: Performance
\item 'Niceness' del proceso: -20
\item Archivo de salida: /dev/null
\end{itemize}